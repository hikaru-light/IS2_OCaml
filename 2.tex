\documentclass{jarticle}

\usepackage{latexsym}

\title{情報科学演習及び実験1(Ocamlの基礎)}
\author{IS2 6316047 出納 光}
\begin{document}

\maketitle

\section{dellt}
\begin{verbatim}
# dellt;;
- : int -> ’a list -> ’a list = <fun> 
# dellt 0 [1;2;3;4];;
- : int list = [1; 2; 3; 4]
# dellt 1 [1;2;3;4];;
- : int list = [2; 3; 4]
# dellt 2 [1;2;3;4];;
- : int list = [3; 4]
# dellt 3 [1;2;3;4];;
- : int list = [4]
# dellt 5 [1;2;3;4];;
- : int list = []
# dellt 3 ["A"; "B"; "C"; "D"; "E"; "F"];;
- : string list = ["D"; "E"; "F"]
# dellt(-2) [1; 2];;
- : Exception: Failure "Error"

1番目の引数の値の分だけ、2番目の引数であるリストの先頭から要素を消す。

1番目の引数を正負で条件分岐させる。
負の値のときは、エラーを返す。
リストから一つずつ要素を取り出し、1番目の引数の値を-1する。
再帰させ、1番目の引数が0になるまで繰り返す。
0になれば、そのリストを返す。
\end{verbatim}

\newpage
\section{dellt2}
\begin{verbatim}
# dellt2;;
- : int -> ’a list -> ’a list = <fun>
# dellt2 1 ["A"; "B"; "C"; "D"; "E"; "F"];;
- : string list = ["B"; "C"; "D"; "E"; "F"]
# dellt2 3 ["A"; "B"; "C"; "D"; "E"; "F"];;
- : string list = ["A"; "B"; "D"; "E"; "F"]
# dellt2 3 [1; 2; 3; 4; 5; 6];;
- : int list = [1; 2; 4; 5; 6]

2番目の引数のリストの先頭から数えて、1番目の引数番目を消す。

1番目の引数を正負で条件分岐させる。
パターンマッチングにより、1以上の値のときはリストの先頭をそのまま付け足し、
値を-1していく。
1になったときは付け足さず、リストの残りを返す。
\end{verbatim}

\newpage
\section{find}
\begin{verbatim}
# find;;    
- : ’a -> ’a list -> bool = <fun> 
# find 3 [1; 2; 3; 4; 5; 6];;    
- : bool = true    
# find 1 [2; 3; 4; 5];;    
- : bool = false

1番目の引数の値が2番目の引数のリスト内に存在かどうか、booleanで判定する。

パターンマッチングにより、リストの先頭を取り出し、その値を1番目の引数と一致しているかで条件分岐させる。
もし違うならば、再帰させ、次の要素を調べる。
\end{verbatim}

\newpage
\section{posl}
\begin{verbatim}
# posl;;    
- : int -> ’a list -> ’a = <fun>    
# posl 3 ["AB"; "C"; "DEF"; "G"; "H"; "IJ"];;    
- : string = "DEF"    
# posl 2 [ 1; 2; 3; 4; 5];;    
- : int = 2    
posl 0 [ 1; 2; 3; 4; 5];;    
Exception: Failure "Not Exist...".

2番目の引数のリストの先頭から、1番目の引数番目を取り出す。

パターンマッチングにより、リストの先頭を取り出し、1番目の引数を-1する。
再帰させ、1になったとき、先頭の要素を取り出す。
\end{verbatim}

\newpage
\section{divlist}
\begin{verbatim}
# divlist;;    
- : int list -> int list -> int list = <fun>
# divlist [12; 8; 9; 7] [2; 4; 3; 7];;
- : int list = [6; 2; 3; 1]

1番目の引数のリストを、2番目の引数のリストで割る。

パターンマッチングにより、それぞれのリストの先頭を取り出し、1番目から2番目を割る。
それを再帰関数と結合させる。
\end{verbatim}

\newpage
\section{mul2list}
\begin{verbatim}
# mul2list;;
- : int list -> int list = <fun>
# mul2list [1; 2];;
- : int list = [2]
# mul2list [1; 2; 3];;
- : int list = [2; 6]
# mul2list [1; 2; 3; 4; 5];;
- : int list = [2; 6; 12; 20]

リストの隣り合った要素を掛け合わせ、それらを新たなリストとする。

パターンマッチングにより、先頭から2つ要素を取り出し、それを掛け合わせる。
2番目の要素をリストの残りを新たに引数として、再帰させる。
\end{verbatim}

\newpage
\section{chglist}
\begin{verbatim}
# chglist;; 
- : ’a * ’a -> ’a list -> ’a list = <fun> 
# chglist ("A", "*") ["1"; "A"; "2"; "B"; "A"; "A"; "3"; "4"];; 
- : string list = ["1"; "*"; "2"; "B"; "*"; "*"; "3"; "4"]

1番目の引数として、置き換える対象と結果を組みで取り、
2番目の引数として、そのリストの中から探し、置き換える。

パターンマッチングにより、リストの先頭を取り出し、
それが組の1番目と等しければ、2番目を結合させて、再帰させる。
等しくなければ、それ自体を結合させる。
\end{verbatim}

\newpage
\section{inslist}
\begin{verbatim}
inslist;; 
- : int -> ’a -> ’a list -> ’a list = <fun> 
# inslist 2 "*" ["A"; "B"; "C"; "D"; "E"];; 
- : string list = ["A"; "*"; "B"; "C"; "D"; "E"] 
# inslist 6 "*" ["A"; "B"; "C"; "D"; "E"];; 
- : string list = ["A"; "B"; "C"; "D"; "E"; "*"] 
# inslist 1 "+" [];; 
- : string list = ["+"] 
# inslist 1 "+" ["A"];; 
- : string list = ["+"; "A"] 
# inslist 0 "+" ["A"];; 
Exception: Failure "Error".

リストに対して、1番目に取った引数の値番目に引数を加える。

パターンマッチングにより、リストの先頭を取り出し、再帰させる。
1番目の引数は再帰のたびに-1しており、1になったとき、2番目の引数も加える。
\end{verbatim}

\newpage
\section{replicate}
\begin{verbatim}
# replicate;; 
- : int -> ’a -> ’a list = <fun> 
# replicate 3 ["A"];; 
- : string list list = [["A"]; ["A"]; ["A"]] 
# replicate 5 "A";; 
- : string list = ["A"; "A"; "A"; "A"; "A"] 
# replicate 3 ["1"; "#"];; 
- : string list list = [["1"; "#"]; ["1"; "#"]; ["1"; "#"]]

1番目の引数の数と同じだけ、2番目の引数をリストに加える。

条件分岐により、1番目の引数が1より大きいときには2番目の引数を加え、1番目の引数を-1して再帰させる。
1になれば、空リストに変えて、終了させる。
\end{verbatim}

\newpage
\section{merge}
\begin{verbatim}
# merge;; 
- : ’a list -> ’a list -> ’a list = <fun> 
# merge [1; 2; 3] [4; 5; 6];; 
- : int list = [1; 4; 2; 5; 3; 6] 
# merge ["A"; "B"] [ "C"; "D"; "EF"; "GH"];; 
- : string list = ["A"; "C"; "B"; "D"; "EF"; "GH"]

2つのリストから、1番目と2番目の順で交互に要素を取り出し、
1つのリストにする。

パターンマッチングにより、それぞれのリストの先頭を取り出し、
1番目、2番目の順で結合させる。
残りを引数として、再帰させ、どちらかのリストが空になった時点で、もう一方のリストを全て返す。
\end{verbatim}

\newpage
\section{inside-length}
\begin{verbatim}
# inside_length;; 
- : 'a list list -> int = <fun> 
# inside_length[[1; 2; 3]; [4; 5]; [6]; [7; 8; 9; 10]];; 
- : int = 10 
# inside_length[["A"; "B"]; [ "C"; "D"]; ["EF"; "GH"]];; 
- : int = 6

リスト内の全ての要素(リストの中のリストの要素もカウントする)を数える。

パターンマッチングにより、リストの先頭のリストを取り出し、
さらに、それにパターンッマッチングを用いることで、そのリスト内の先頭の要素を取り出し、+1する。
再帰させ、リスト内の要素を調べ終わったら、inside_lengthで再帰させ、全てのリストを調べる。
\end{verbatim}

\newpage
\section{concat}
\begin{verbatim}
# concat;; 
- : 'a list list -> 'a list = <fun> 
# concat [[0; 3; 4]; [2]; [0]; [5; 0]];; 
- : int list = [0; 3; 4; 2; 0; 5; 0]

リストの中のリストの要素を取り出し、それを一つのリストにまとめる。

パターンマッチングにより、リストの先頭のリストを取り出し、
さらに、それにパターンッマッチングを用いることで、そのリスト内の先頭の要素を取り出し、新たに用意したリストに加える。。
再帰させ、リスト内の要素を結合し終わったら、concatで再帰させ、全てのリストを結合させる。
\end{verbatim}

\newpage
\section{assoc}
\begin{verbatim}
# assoc;; 
- : ’a -> (’a * ’a) list -> ’a = <fun> 
# assoc 33 [(3,4); (33,5); (11,2); (55,1)];; 
- : int = 5 
# assoc 2 [(3,4); (33,5); (11,2); (55,1)];; 
- : int = 11 
# assoc "03" [("Kyoto", "075"); ("Osaka", "06"); ("Tokyo", "03")];; 
- : string = "Tokyo" 
# assoc "Kyoto" [("Kyoto", "075"); ("Osaka", "06"); ("Tokyo", "03")];; 
- : string = "075" 
# assoc 6 [(3,4); (33,5); (11,2); (55,1)];; 
Exception: Failure "Not found...".

リストの中の組を調べ、1番目の引数にとっていたものが存在すれば、
その組の一方を返す。
どこにも無ければ、エラーを返す。

パターンマッチングにより、リストの先頭の組を取り出し、
条件分岐により、手前の要素が1番目の引数と同じならば、後ろの要素を返す。
後ろの要素が1番目の引数を同じならば、手前の要素を返す。
再帰させ、全てのリストを調べる。
全てのリストを調べ、該当するものが無ければ、エラーを返す。
\end{verbatim}

\newpage
\section{muximum}
\begin{verbatim}
# maximum;; 
- : 'a list -> 'a = <fun> 
# maximum [3; 2; 5; 1];; 
- : int = 5 
# maximum ["abc"; "sdf"];; 
- : string = "sdf" 
# maximum [];; 
Exception: Failure "Error".

リストの中から、最大の値を見つける。

パターンマッチングにより、先頭の2つの要素を取り出し条件分岐により、大小を比較する。
大きい方とリストの残りを再帰させる。
要素が一つの場合は、それを返す。
無い場合は、エラーを返す。
\end{verbatim}

\newpage
\section{index}
\begin{verbatim}
# index;;    
- : ’a list -> ’a -> int = <fun> 
# index [21; 2; 31; 1] 21;;    
- : int = 1    
# index [21; 2; 31; 1] 2;;    
- : int = 2    
# index [21; 2; 31; 1] 3;;    
- : int = -1    
# index [] 3;;    
- : int = -1

2番目の引数と同じ値を1番目の引数のリストから探し、あれば先頭から数えて何番目かを表す。
無ければ、-1を返す。

パターンマッチングにより、先頭の要素を取り出し、それが2番目の引数と等しければ、再帰させた数-1を返す。再帰させ、リストの中が無くなれば、-1を返す。
\end{verbatim}

\section{keirosu}
\begin{verbatim}
碁盤目状の道路があるとします.始点から終点までの経路の数を求めてみましょう.ただし,始点から終点までの経路は最短経路のみとします.すなわち,進行方向は右方向または上方向に限られ,左方向または下方向に進むことできないことにします.

\end{verbatim}

\end{document}

